
%%%%%%%%%%%%%%%%%%%%%%%%%%%%%%%%%%%%%%%%%%%%%%%%%%%%%%%%%%%%%%%%%%%
%%
%% Mahesh Kumar's Resume based on res_yy.sty 
%%
%%%%%%%%%%%%%%%%%%%%%%%%%%%%%%%%%%%%%%%%%%%%%%%%%%%%%%%%%%%%%%%%%%%



%%
%% The following code sets up the document formatting
%%

%this assumes that res_yy.sty is in some path
\documentstyle[hyperref, margin, line]{res_yy}

\hypersetup{backref,pdfpagemode=Full,colorlinks=true,backref}

\addtolength{\oddsidemargin}{-0.45in}
\addtolength{\voffset}{-0.30in}
\addtolength{\textwidth}{1.00in} \addtolength{\textheight}{1.50in}

\renewcommand{\namefont}{\LARGE\emph}



%%
%% The following code defines some macros for terms which have raised font
%% (ie 4\fourth would result 4th with the 'th' raised (superscripted)
%%

\def\Cplusplus{{\rm C\raise.5ex\hbox{\small ++}}}
\def\CSharp{{\rm C\raise.5ex\hbox{\small \#}}}
% 'st' 'nd' 'rd' 'th' superscripts for numbers
\def\first{{\raise.5ex\hbox{\small st}}}
\def\second{{\raise.5ex\hbox{\small nd}}}
\def\third{{\raise.5ex\hbox{\small rd}}}
\def\fourth{{\raise.5ex\hbox{\small th}}}



%%
%% starting the actual document
%%

\begin{document}

%the name in big fonts at the top of resume
%this is left aligned
\name{Mahesh Kumar Vadla}

%this is right aligned
\address{
Email: mahesh4gnu@gmail.com  \ \ \ \ \ Mobile: +91-9491884300
}

\begin{resume}



%%
%% This section of code is inelegant, but I'm too lazy to fix it
%%

\section{\textsc{Summary}}
Self-motivated individual with over 2 years of experience in Systems
Engineering and programming. Esteemed to work on Free/Open Source
Software. Opted to work in any development stream.  \\
 
%%
%% the meat of the resume starts now
%%

\begin{formatb}
  \employer{l}\title{r}\\
  \location{l}\dates{r}\\
  \body\\
\end{formatb}

\section{\textsc{Professional Experience}}
\employer{\textbf{International Institute of Information Technology}}
\title{\em \textbf{Project Engineer}}
\location{Hyderabad, Andhra Pradesh}
\dates{\em \textbf{2011 July till date}\vspace*{0.3cm}}
\begin{position}
\textbf{Description}: Virtual Labs is a mission mode project funded by
Ministry of Human Resource Development(MHRD), India. Professors and
students from various institutions in India develop labs based on
various courses. The goal of our team is to give engineering services
to lab developers like development, deployment and distribution of
virtual labs.\\

\textbf{Role}: Maintaining LDAP,Virtual Machines and development
servers, deploying the labs, implementing CMS(drupal) and
wiki(mediawiki) tools for different purposes, Packaging the labs,
implementing the LiveCD distribution model and Virtual-labs Web
Analytics maintenance.\\

\textbf{Contributions}:
\begin{itemize}
\item Creating and maintaining the accounts for the developers of the
  Labs using LDAP and Redmine
\item Helping out the Lab Developers to deploy their labs in our
  process model
\item Packaging the labs as debian binary packages (.deb)
\item Created the LiveCD for the virtual-labs (hosted by
  http://virtual-labs.ac.in)
\item Installed the web analytics tool ``Piwik'' and maintained the same
  for the virtual-labs web pages and generated the reports.
\item Installed the wiki documentation tool ``mediawiki'' and
  configured the LDAP authentication for the same.
\item Worked with Drupal and Joomla for Virtual Labs engineering
  purposes.
\item Training the lab developers and staff of Virtual Labs on using
  version control systems (Bazaar,GIT and SVN) for their labs
  development

\end{itemize}
\textbf{Technologies Used}: LDAP, Redmine, SSH, SVN, PHP, Shell
Scripting, Debian GNU Tools and Piwik Web Analytics Tool.
\end{position}
\\
\employer{\textbf{Innominds Software Pvt. Ltd.}}
\title{\em \textbf{Trainee Software Engineer}}
\location{Hyderabad, Andhra Pradesh}
\dates{\em \textbf{2010 December to 2011 June}\vspace*{0.3cm}}
\begin{position}
\textbf{Description}: Innominds Software is a leading Outsourced
Software Product Engineering Services company which provides the
engineering services for the Mobile and Telecom Technology, Enterprise
Software and Testing the products.\\
\textbf{Role}: Trained on Webkit based browser for BREW mobile
platform, resolving the issues on the same browser and the Opensource
Webkit under GTK+ and submitted the patches to Webkit Opensource.\\
\textbf{Contributions}: 
\begin{itemize}
\item Resolved some issues and bugs in the webkit based BMP Browser
  developed by Innominds
\item Resolved some opensource open issues on rendering of the Webkit
  engine under GTK+ and Qt.
\end{itemize}
\textbf{Technologies used}: C, BREW Mobile Platform, GIT Version
Control  
\end{position}
\\

%%
%% Educational Qualifications
%%

\section{\textsc{Education}}

\textbf{C.V.S.R. College Of Engineering, Hyderabad} (Affi. to \textbf{JNTU-H})\hfill 2006 - 2010 \\
B.Tech in Computer Science \& Engineering\hfill \\
\newline
\textbf{Narayana Junior College (for Boys), Narayanaguda, Hyderabad} \hfill 2005 - 2006 \\ 
Board Of Intermediate Education (II year) \\
\textbf{Gnanadeep Junior College, Kamareddy} \hfill 2004 - 2005\\
Board Of Intermediate Education (I year) \\
\newline
\textbf{Nagarjuna High School, Kamareddy} \hfill 2003 - 2004 \\
Secondary School Certificate
\\


%%
%% Academic Projects
%%

\section{\textsc{Academic Projects}}
\employer{\textbf{Pagination Application}}
\title{\em \textbf{Main Project}}
\location{Swecha (a non-profit organisation), Andhra Pradesh}
\dates{\em \textbf{3 Months}\vspace*{0.3cm}}
\begin{position}
\textbf{Description}: This project is to provide an application for
the DTP operators for the pagination process. This is a Free \& Open
source Software which is built by using the GTK+ Libraries.\\
\textbf{Role}: My module in this project is, File Format. i.e, how the
data of the documents of the application can be stored/retrieved
from/to that application.\\
\textbf{Technologies Used}: XML parsers with Python.\\
\end{position}
\employer{\textbf{Building GNU/Linux Operating System From Scratch}}
\title{\em \textbf{Mini Project}}
\location{Swecha (a non-profit organisation), Andhra Pradesh}
\dates{\em \textbf{1 Month}\vspace*{0.3cm}}
\begin{position}
\textbf{Description}: This project is subjected to give a basic idea
of how to build our own operating system and how  to  customize it
according to our needs. In this project, I had also provided that how
to make a live environment of an operating system.\\
\textbf{Role}: Single-army Project\\
\textbf{Technologies Used}: Basic Unix commands and integration of
Linux kernel.\\
\end{position}


%%
%% We use the same formatting for projects as for work experience
%% Shown below is the formatting used previously
%%
%%  \begin{formatb}
%%    \employer{l}\title{r}\\
%%    \location{l}\dates{r}\\
%%    \body\\
%%  \end{formatb}
%%
%% 
%%  Note that \location is now being used for non-location information
%%


\begin{formatb}
  \employer{l}\dates{r}\\
  \body\\
\end{formatb}

\section{\textsc{Technical Skills}}

\employer{\textbf{Debian GNU/Linux and derivatives}}
\dates{}
\begin{position}
Knowledge in administering the GNU/Linux operating systems (Debian and
its derivatives), basic knowledge on installing and maintaining LDAP
Servers, Apache Server, Mysql server.Created custom binary packages
for debian (*.deb) for internal software at job locations. Scheduling
tasks using ``cron''. Custom built Debian based distributions for
installing at client and office locations.
\end{position}

\employer{\textbf{Virtualization}}
\dates{}
\begin{position}
Implementing ``KVM'' for fully and para virtualization methods. Can
implement the ``KVM'' and Container based virtualization (``OpenVZ'')
using the ``Proxmox''. Backup and migrate virtual servers in live and
offline image based replications. Can troubleshoot containers and
fully virtualized images by mounting and using ``chroot'' based
techniques. 

\end{position}
 
\employer{\textbf{Web Technologies/Web Applications}}
\dates{}
\begin{position}
Intermediate knowledge in HMTL, CSS, JavaScript, Jquery, XML and 
PHP. Knowledge on using the Content Management Systems like Drupal,
web applications like Django and Ruby on Rails (RoR). Maintained
Project Management Portal (Redmine). Maintained CGI based web
administration panels(Easypush from deepOfix) for the LDAP
accounts. Proficient knowledge on installing and maintaining the web
analytics using ``Piwik''.

\end{position}

\employer{\textbf{Programming/Scripting Languages}}
\dates{}
\begin{position}
Work experience in C programming language, knowledge on \Cplusplus
and Java(Core) and the basic knowledge on Python, Perl, Ruby and Shell
Scripting. Ability to debug (using GDB), understand and analyse the
code comfortably.
\end{position}

\employer{\textbf{Database Technologies}}
\dates{}
\begin{position}
Comfort in using the ``Mysql'' in textual and graphical
mode. Administering the ``Mysql Server'' and usage of ``SQL''
language. Basic idea on Object Oriented DBMS like ``PostgreSQL'' and
Document based DBMS like ``CouchDB''.  
\end{position}

\employer{\textbf{Editors/Tools}}
\dates{}
\begin{position}
Experience in using the IDEs for programming like ``Eclipse'' for Java,
``Anjuta'' and ``Visual Studio'' for C/\Cplusplus. ``BlueFish'' for web
programming. ``Emacs'' for standard text editing, ``Org-mode'' for
personal organizing, project/task management, publishing reports,
documentation etc. \LaTeX \ for documentation. ``dot(Graphviz)'' for
diagramatic representation and ``Umbrella'' for project design
diagrams. ``GIMP'' and ``Inkscape'' for image manipulation. The
general stack of interests and desktop usage include ``Debian or
Ubuntu GNU/Linux, LXDE/Gnome Desktop Environment, Emacs,
Terminator(terminal program), Firefox Web Browser'' in all. Version
control using ``svn'' and ``git''.\\
\end{position} 

%%
%% This section could also use more formatting, but looks ok, as is
%%

%\section{\textsc{Qualifications}}

%\emph{Programming Languages}: \Cplusplus, \CSharp, Cg, HLSL, ARB assembly, SML, OCaML, PHP, MySQL, Java, Python, Perl, MIPS assembly

%\emph{Libraries and Tools}: Vim, STL, DirectX, OpenGL, \LaTeX, GIMP, Adobe Suite, Macromedia Suite, MatLab, Mathematica, Microsoft Visual Studio, GCC, GDB


%%
%% Note that we're redefining the formatting
%% We only have one row of information now, instead of two
%%

\section{\textsc{Activities}}

\begin{formatb}
  \employer{l}\dates{r}\\
  \body\\
\end{formatb}

\employer{\textbf{Swecha}}
\dates{2007-2010}
\begin{position}
An active volunteer for Swecha(http://swecha.org), an N.G.O working on
Localization of GNU/ Linux and advocacy of Free Software in
academia. Participated in many advocacy events and technical camps
spreading philosophical and technical awareness on FOSS. Volunteered
and participated in 1st state conference of Swecha held in 2010 and
Attended for national conferences on Free Software held in 2007 at
Hyderabad and 2010 at Bangalore.

\end{position}

\employer{\textbf{Drupal Hyderabad}}
\dates{2009-2010}
\begin{position}
Volunteer at ``Drupal Hyderabad'' a Drupal based community based in
Hyderabad working for promoting and educating Drupal based portals
across companies and academia.\\ 
\end{position}


\section{\textsc{Personal Details}}


\textbf{Father's Name} : Vadla Vishwanatham

\textbf{Gender} : Male

\textbf{Date Of Birth} : 23\third April, 1989

\textbf{Languages known} : English, Telugu, Hindi

\employer{\textbf{Email IDs/Public Profile URLs} :}
\dates{}
\begin{position}
\emph{mahesh4gnu@gmail.com}\\
\emph{mahesh@virtual-labs.ac.in}\\
Linkedin Profile:\emph{ http://in.linkedin.com/in/maheshkumarvadla}\\
Diaspora Profile:\emph{ https://joindiaspora.com/u/maheshkumar}
\end{position}

\employer{\textbf{Address (Permanent)} :}
\dates{}

\section{\textsc{Credits}}

\begin{formatb}
  \employer{l}\dates{r}\\
  \body\\
\end{formatb}

\employer{This resume is prepared in \LaTeX \ using res\_yy.sty and
  typed in Emacs text editor}
\dates{}
\begin{position}
\end{position}


\section{\textsc{Declaration}}

\begin{formatb}
  \employer{l}\dates{r}\\
  \body\\
\end{formatb}

I hereby declare that the above-mentioned information is
  correct to my knowledge and I bear the responsibility for the
  correctness of the above-mentioned particulars. \\\\\\\\\\

\section{\textsc{Place}:}

\section{\textsc{Date}:}
\begin{formatb}
  \employer{l}\dates{r}\\
  \body\\
\end{formatb}
\employer{}
\dates{\textbf{(Mahesh Kumar)}}
\begin{position}
\end{position}
%%
%% Nothing special here, just a normal table
%%

%\section{\textsc{Course Work}}
%  \begin{tabular}{lllll}
%  Information Networks   & \ \ & Machine Learning    & \ \ & Theory of Computation \\ 
%  Computer Graphics      & \ \ & Machine Vision      & \ \ & Programming Languages \\
%  Software Engineering   & \ \ & Algorithms          & \ \ & Artificial Intelligence     \\
%  Operating Systems      & \ \ & Databases           & \ \ & Computer Architecture \\
%  Numerical Methods      & \ \ & Graph Theory        & \ \ & Differential Equations      \\
%  Probability Theory     & \ \ & Number Theory       & \ \ & Differential Geometry       \\
%  Advanced Calculus      & \ \ & Abstract Algebra    & \ \ & Advanced Combinatorics   \\
%  \end{tabular}


\end{resume}
\end{document}
